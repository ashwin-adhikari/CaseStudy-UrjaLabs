
\chapter{MANPOWER MANAGEMENT}

\section{Personnel Policies}
It's about the ground rule that company has about making decision about manpower and personnel such as organization of employees decision-making role HR policies etc...

The CEO manages it all.
\section{Manpower Planning}
provide projects to those who are learning from URJA TECH ACADEMY

Search on LinkedIn
\section{Recruitment and selection of manpower}
\subsection{Based on HR Policy}

\subsection{Interview Process Overview}

Our comprehensive interview process is designed to evaluate candidates thoroughly across multiple dimensions, ensuring they are a good fit for both the role and our organizational culture. The process consists of three distinct phases:
\begin{itemize}
\item Phase 1: General Interview

The first phase of the interview process is the General Interview. This initial interview serves multiple purposes:

     \begin{itemize}

    \item Introduction and Overview:
        The candidate is introduced to our company's mission, vision, values, and culture. This helps in setting the context for subsequent interviews and ensures the candidate understands what we stand for.

    \item Resume Review:
        The interviewers review the candidate’s resume in detail, discussing their previous work experience, education, and any other relevant information. This helps in verifying the credentials and understanding the candidate’s background.

    \item General Competency Assessment:
        The candidate is assessed on general competencies and soft skills such as communication, problem-solving, teamwork, and adaptability. Behavioral questions are often used to understand how the candidate has handled various situations in the past.
   \end{itemize} 
\item Phase 2: Field-Specific Interview

The second phase focuses on the technical and field-specific skills required for the role:
    \begin{itemize}
    \item Technical Knowledge Assessment:
        Candidates are evaluated on their technical expertise and field-specific knowledge. This may include theoretical questions, practical problems, case studies, or technical tasks relevant to the position they are applying for.

    \item Skills Verification:
        The candidate’s ability to apply their knowledge in real-world scenarios is tested. This may involve situational questions or live demonstrations of skills, depending on the nature of the job.

    \item Discussion with Subject Matter Experts:
        The candidate will have an in-depth discussion with senior team members or subject matter experts who can assess the candidate's proficiency in the specific field. This ensures that the candidate has the necessary expertise to excel in the role.
    \end{itemize} 
\item Phase 3: Solo Interview (Perspective and Attitude-Based)

The final phase is the Solo Interview, which focuses on the candidate's personal attributes and cultural fit:
    \begin{itemize}
    \item Perspective Evaluation:
        Interviewers assess the candidate’s perspectives on various work-related and ethical scenarios. This helps in understanding the candidate's thought process, decision-making abilities, and how they align with the company's values.

    \item Attitude and Cultural Fit:
        The candidate's attitude towards work, their motivation, and their alignment with the company culture are evaluated. Questions may be centered around hypothetical situations to gauge the candidate's reactions and approach.

    \item Self-Reflection and Future Goals:
        The candidate is encouraged to reflect on their career journey, strengths, weaknesses, and future aspirations. This helps in understanding their long-term goals and how they envision their growth within the company.
    \end{itemize}     
\end{itemize}

In this way, the selection of employee or a new manpower is done through these three phases of interview. It may seem the second round is mostly important for many peoples, the third round of interview is actually important to land the job.
\section{Training and development of manpower}
\subsection{Reach Out to the Tech Masters}


Engaging with experienced professionals in the technology field is a vital part of training. These "tech masters" can be mentors, trainers, or industry experts who provide valuable insights and hands-on knowledge.

Benefits:
\begin{itemize}
    

  \item  Expertise Access: Trainees gain access to a wealth of knowledge and real-world experience.
  \item Personalized Learning: Mentors can tailor training to individual learning needs and styles.
  \item Networking: Building relationships with established professionals can lead to future opportunities and collaborations.
\end{itemize}
Implementation:
\begin{itemize}
  \item   Mentorship Programs: Pair trainees with seasoned professionals for one-on-one guidance.
    \item Workshops and Seminars: Organize sessions where tech masters share their expertise on specific topics.
    \item Guest Lectures: Invite experts to give talks or conduct webinars on relevant subjects.
\end{itemize}
\subsection{Online Resources}


Leveraging online resources is essential for modern training programs. These resources include online courses, tutorials, webinars, forums, and e-books that provide flexibility and a wide range of learning materials.

Benefits:
\begin{itemize}
   \item Flexibility: Learners can access materials at their own pace and convenience.
    \item Diverse Content: A vast array of topics and formats cater to different learning preferences.
    \item Cost-Effective: Many high-quality resources are available for free or at a low cost.
\end{itemize}
Implementation:
\begin{itemize}
    \item E-Learning Platforms: Utilize platforms like Coursera, Udemy, LinkedIn Learning, or Khan Academy for structured courses.
    \item Webinars and Tutorials: Encourage participation in live webinars and video tutorials on platforms like YouTube or specific tech sites.
    \item Discussion Forums: Promote engagement in forums such as Stack Overflow, Reddit, or specialized tech communities to solve problems and share knowledge.
\end{itemize}
\subsection{Providing Work that hones their skills}


Practical experience is crucial for reinforcing learning. "Work, work, work" emphasizes the importance of hands-on practice, projects, and real-world application of skills.

Benefits:
\begin{itemize}
    \item Skill Application: Practical tasks help consolidate theoretical knowledge and develop problem-solving skills.
    \item Experience Building: Real-world projects provide experience that is valuable for future employment or career advancement.
    \item Confidence Boosting: Completing projects and solving problems builds confidence in one’s abilities.
\end{itemize}
Implementation:
\begin{itemize}
    \item Project-Based Learning: Assign projects that require applying learned concepts to create tangible outcomes.
    \item Internships and Practicums: Facilitate opportunities for trainees to work in real-world environments through internships or practicum programs.
    \item Hackathons and Competitions: Encourage participation in hackathons, coding challenges, or other competitions to foster a competitive and collaborative spirit.
\end{itemize}    
    
\section{Job evaluation and Merit System}
For the job evaluation, the company has decided to use following factors into consideration for the employee.
\begin{itemize}
    \item Rewards
    \begin{itemize}
    \item Incentives: Monetary (bonuses, raises) and non-monetary (recognition, extra time off).
    \item Benefits: Boosts motivation, retention, and productivity.
    \item Implementation: Performance bonuses, recognition programs, and additional perks.
\end{itemize}
    \item Self-Assessment
    \begin{itemize}
    \item Purpose: Employees evaluate their own performance.
    \item Benefits: Enhances self-awareness, ownership, and continuous improvement.
    \item Implementation: Use structured templates, regular intervals, and goal setting.
\end{itemize}
    \item Evaluation through Performance
    \begin{itemize}
    \item Overview: Performance reviews with face-to-face feedback.
    \item Benefits: Provides clear feedback, strengthens relationships, aligns goals.
    \item Implementation: Use performance metrics, conduct review meetings, document evaluations.
\end{itemize}
    \item 1 to 1 talk
    \begin{itemize}
    \item Purpose: Personalized feedback and support.
    \item Benefits: Builds trust, offers tailored advice, addresses issues promptly.
    \item Implementation: Regular check-ins, open dialogue, develop action plans.
\end{itemize}
\end{itemize}
As the company is currently in stagnant stages, they had chosen not take a bold step to put themselves in bigger stage. Putting themselves in a position not take consideration of merit system or rating for the employee as it might not be effective.
\section{Removing of manpower}
Sometimes it requires a few bold actions for a company to work in a very sustainable manner without destroying the ecosystem the company has provided over the past few years. So removing some of the employee re always a option for top of the management.

There are few considerations given while removing manpower.
\begin{itemize}
    \item Intern:
    \begin{itemize}
        \item They may get remove with immediate effect, without consideration.
        \item Friends are allowed to intern there as it may disturb the ecosystem and prevention from groupism
    \end{itemize}    
    \item Beginner Member:
    \begin{itemize}
        \item They will be given a few weeks notice before they are removed from the job.
    \end{itemize}
    \item Senior Member:
    \begin{itemize}
        \item They will be given two or three months notice before they are removed from the job.
        \item They will also be compensated with some money about few months salary.
    \end{itemize}
    
\end{itemize}

 
